\documentclass[12pt,conference,onecolumn]{IEEEtran}

\title{Naval architecture and hull design}
\author{%
\IEEEauthorblockN{Kevin Tomazic}\IEEEauthorblockA{Science \& Engineering\\Manalapan High School\\Englishtown, NJ\\425ktomazic@frhsd.com}}
\date{June 16, 2025}

\newcommand{\keywords}{naval architecture, CAD, computer aided design, Fusion 360, hull design, stability, buoyancy, resistance and powering, sailboat, sailbot, autonomous sailbot, 3D printing, additive manufacturing}

\usepackage{hyperref}
\makeatletter
\AtBeginDocument{
\hypersetup{%
pdftitle={\@title},
pdfauthor={Kevin Tomazic},
pdfkeywords={\keywords}}}
\makeatother

\usepackage{siunitx}

\begin{document}
\maketitle 

\begin{abstract}
Naval architecture is the study of ship design, focusing on overall balanced design considering the functional requirements, with special focus on the hull, including buoyancy; centers, moments, and stability; structural integrity; resistance and powering, etc. I examined the design of a small \qty{1}{\meter} LOA hull intended for a model or autonomous sailing craft. To design the hull and obtain volumes and centers for stability calculations, I used Autodesk Fusion. To physically construct the hull, I used additive manufacturing (3D printing), using several hull sections in order to satisfy manufacturability constraints due to bed length of the tooling.
\end{abstract}

\begin{IEEEkeywords}
\keywords
\end{IEEEkeywords}

\end{document}