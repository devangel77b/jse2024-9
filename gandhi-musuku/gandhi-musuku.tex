\documentclass[12pt,conference,onecolumn]{IEEEtran}

\title{{5G} {E}dge cloud application}
\author{\IEEEauthorblockN{Dilan Gandhi}\IEEEauthorblockA{Science \& Engineering\\Manalapan High School\\Englishtown, NJ\\525dgandhi@frhsd.com}\and
\IEEEauthorblockN{Shreyas Musuku}\IEEEauthorblockA{Science \& Engineering\\Manalapan High School\\Englishtown, NJ\\425smusuku@frhsd.com
}}
\date{June 16, 2025}

\newcommand{\keywords}{5G Edge computing,  vehicle-to-vehicle communication, autonomosu vehicles, real-time data, MQTT, low latency, traffic safety, OBD-II emulator, Docker deployment, collision avoidance, connected vehicles, intelligent transportation systems}

\usepackage{hyperref}
\makeatletter
\AtBeginDocument{
\hypersetup{%
pdftitle={\@title},
pdfauthor={Dilan Gandhi and Shreyas Musuku},
pdfkeywords={\keywords}}}
\makeatother

\begin{document}
\maketitle 

\begin{abstract}
This project investigates the potential of 5G Edge computing to improve vehicular safety through ultra-low latency communication and real-time data processing. With human error accounting for the majority of traffic accidents, our system leverages the proximity of edge servers to 5G base stations to transmit critical vehicle telemetry between nearby cars using MQTT protocols and Docker-based deployment. We designed and tested two key safety scenarios—following vehicle and head-on collision detection—demonstrating reliable, low-latency performance under simulated conditions using OBD-II emulators. Our work also includes a documented framework for scalable deployment and real-time data streaming. Looking forward, we aim to integrate machine learning for predictive maneuvers such as lane changes and environmental adaptation, enhancing the system's robustness in complex traffic environments.
\end{abstract}

\begin{IEEEkeywords}
\keywords
\end{IEEEkeywords}

\end{document}