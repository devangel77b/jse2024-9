\documentclass[12pt,conference,onecolumn]{IEEEtran}

\title{Algorithms for analyzing risk of aneurysms}
\author{%
\IEEEauthorblockN{Eshan Handique}\IEEEauthorblockA{Science \& Engineering\\Manalapan High School\\Englishtown, NJ\\425ehandique@frhsd.com} \and 
\IEEEauthorblockN{Justin Hammer}\IEEEauthorblockA{Science \& Engineering\\Manalapan High School\\Englishtown, NJ\\425jhammer@frhsd.com}}
\date{January 28, 2025}

\newcommand{\keywords}{aneurysm, risk factor analysis, detection, SQL}

\usepackage{hyperref}
\makeatletter
\AtBeginDocument{
\hypersetup{%
pdftitle={\@title},
pdfauthor={Eshan Handique and Justin Hammer},
pdfkeywords={\keywords}}}
\makeatother

\begin{document}
\maketitle 

\begin{abstract}
Aneurysms are bulges in the blood vessel wall. If they rupture and break, they can cause many harmful effects and even lead to death. The primary causes of aneurysms include high blood pressure, high cholesterol, and genetic disorders. Still, one can never predict the growth of an aneurysm until it has already manifested in the body. There are currently ways to image an aneurysm within a body to confirm the presence of one, but there is no way to predict outright whether or not someone will grow an aneurysm. First, we simulated an aneurysm in real life to learn about the main factors contributing to an aneurysm's growth. We aim to create a program that would analyze whether or not the user is at risk of developing an aneurysm in the future. This will be done by taking different metrics from a database of aneurysm patients and comparing them to data from the user, discerning whether or not the user’s data correlates with the patients. At this moment, the SQL query and database are complete but the comparison program is still in the works.
\end{abstract}

\begin{IEEEkeywords}
\keywords
\end{IEEEkeywords}

\end{document}