\documentclass[12pt,conference,onecolumn]{IEEEtran}

\title{Automatic detection of microexpressions using an image classification model}
\author{%
\IEEEauthorblockN{Eshan Handique}\IEEEauthorblockA{Science \& Engineering\\Manalapan High School\\Englishtown, NJ\\425ehandique@frhsd.com} \and 
\IEEEauthorblockN{Nathan Martin}\IEEEauthorblockA{Science \& Engineering\\Manalapan High School\\Englishtown, NJ\\425nmartin@frhsd.com}}
\date{January 28, 2025}

\newcommand{\keywords}{polygraph, lie detection, microexpressions, image classification}

\usepackage{hyperref}
\makeatletter
\AtBeginDocument{
\hypersetup{%
pdftitle={\@title},
pdfauthor={Eshan Handique and Nathan Martin},
pdfkeywords={\keywords}}}
\makeatother

\begin{document}
\maketitle 

\begin{abstract}
Polygraph tests determine if someone is lying by comparing differences in their biometric data with their responses to a set of questions. However, this method of lie detection is not very effective and is open to misuse for intimidation and abuse. Because of this, we aimed to create a new functional form of lie detection. This was done by analyzing one’s microexpressions: the uncontrollable expressions that appear on someone’s face for a short amount of time, expressing their true feelings. These microexpressions are thought to give a more truthful indication of the emotional state of the subject. We used an image classification model to automatically detect microexpressions on someone's face in a live feed, allowing us to decide whether or not they are lying based on their answer in relation to their microexpressions. We will continue to refine the model to try to get the lie detector as accurate as possible in the future.
\end{abstract}

\begin{IEEEkeywords}
\keywords
\end{IEEEkeywords}

\end{document}