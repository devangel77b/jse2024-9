\documentclass[12pt,conference,onecolumn]{IEEEtran}

\title{Ultrasonic device for rapid air bubble removal in automotive coolant systems: the saga continues}
\author{%
\IEEEauthorblockN{Anirudh Khanna}\IEEEauthorblockA{Science \& Engineering\\Manalapan High School\\Englishtown, NJ\\425akhanna@frhsd.com} \and 
\IEEEauthorblockN{Nareshsanjay Muthukumar}\IEEEauthorblockA{Science \& Engineering\\Manalapan High School\\Englishtown, NJ\\425nmuthukumar@frhsd.com}}
\date{June 16, 2025}

\newcommand{\keywords}{ultrasound, bubbles, automotive, coolant system, degas, feed-and-bleed, fluid system, noncondensable gases, air-intrusion, piping system}

\usepackage{siunitx}
\usepackage{hyperref}
\makeatletter
\AtBeginDocument{
\hypersetup{%
pdftitle={\@title},
pdfauthor={Anirudh Khanna and Nareshsanjay Muthukumar},
pdfkeywords={\keywords}}}
\makeatother

\begin{document}
\maketitle 

\begin{abstract}
The efficiency and longevity of automotive engines are heavily influenced by the proper operation of cooling systems, which can be compromised by the presence of air bubbles. Traditional methods of removing air bubbles, such as system bleeding, are time-consuming and often ineffective. This project introduces an ultrasonic device designed to rapidly and efficiently eliminate air bubbles from automotive coolant systems. Using a 555 timer chip to create an ultrasonic wave, and several forms of amplification, the device powers an ultrasonic transducer that emits sound waves to promote the coalescence and removal of entrained gases within a coolant system. Preliminary tests show that this ultrasonic method is faster and more efficient than conventional techniques while ensuring compatibility with existing automotive systems. The device improves engine efficiency by enhancing coolant flow and reducing the risk of overheating due to trapped air and foaming. Potential applications include faster maintenance, reduced labor costs, and enhanced system performance. This device could be applicable to a wider range of fields, such as milk pasteurization, oil rigging, and material sciences.
\end{abstract}

\begin{IEEEkeywords}
\keywords
\end{IEEEkeywords}

\end{document}