\documentclass[12pt,conference,onecolumn]{IEEEtran}

\title{Girl in Space Club internship}
\author{%
\IEEEauthorblockN{Victoria Collemi}\IEEEauthorblockA{Science \& Engineering\\Manalapan High School\\Englishtown, NJ\\425vcollemi@frhsd.com} \and 
\IEEEauthorblockN{Cameron Karabin}\IEEEauthorblockA{Science \& Engineering\\Manalapan High School\\Englishtown, NJ\\425ckarabin@frhsd.com} \and
\IEEEauthorblockN{Sabrina Thompson}\IEEEauthorblockA{Girl in Space Club\\sabrina.thompson@girlinspaceclub.com}}
\date{January 28, 2025}

\newcommand{\keywords}{Girl in Space Club, web design, STEM education, flight suit, HTML, CSS, JavaScript, internship, user interface, user experience, UX design}

\usepackage{hyperref}
\makeatletter
\AtBeginDocument{
\hypersetup{%
pdftitle={\@title},
pdfauthor={Victoria Collemi, Cameron Karabin, and Sabrina Thompson},
pdfkeywords={\keywords}}}
\makeatother

\begin{document}
\maketitle 

\begin{abstract}
Girl in Space Club is a clothing brand established by NASA engineer Sabrina Thompson that produces astronaut-grade and fashion flight suits for the analog astronaut (people who participate in crewed simulated space missions on Earth), female astronaut candidates, and STEM enthusiast communities. Since the corporate mission is to make STEM fun, fashionable, and creative, we were tasked with designing a ``Build Your Own Flight Suit'' app that enables customers to individualize their flight suits. To develop the website's functionality and aesthetics, we employed the web development languages of HTML, CSS, and JavaScript. The foundation for our website layout and ``flow'' was based on Ms. Thompson’s design philosophy of ``putting the user’s experience first''. During our presentation, we will present the (1) user engagement principles and logic behind our website structure; (2) the evolution of the website's GUI (graphical user interface) and backend code; and (3) a preview of the website. We will also discuss how we intend to measure user engagement with our website to evaluate its effectiveness.
\end{abstract}

\begin{IEEEkeywords}
\keywords
\end{IEEEkeywords}

\end{document}