\documentclass[12pt,conference,onecolumn]{IEEEtran}

\title{Implementing {S}imultaneous {L}ocalization and {M}apping ({SLAM}) onto a mobile robot}
\author{\IEEEauthorblockN{Saketh Ayyagari}\IEEEauthorblockA{Science \& Engineering\\Manalapan High School\\Englishtown, NJ\\425sayyagari@frhsd.com}}
\date{June 16, 2025}

\newcommand{\keywords}{localization, mapping, SLAM, robot, ROS2, mobile robot, Gazebo, IMU, wheel encoders, LIDAR}

\usepackage{hyperref}
\makeatletter
\AtBeginDocument{
\hypersetup{%
pdftitle={\@title},
pdfauthor={Saketh Ayyagari},
pdfkeywords={\keywords}}}
\makeatother


\usepackage{siunitx}

\begin{document}
\maketitle 

\begin{abstract}
A common problem in autonomous navigation of robots is the ability to both map and consistently know its location in an unknown environment, more commonly known as the Simultaneous Localization and Mapping (SLAM) problem. Localization is a robot's ability to estimate its position and orientation with respect to the world given a predetermined map of an environment, and mapping is its ability to construct an accurate map, assuming it knows its pose (position and orientation). This semester, I focused on implementing a SLAM algorithm in both a simulated and a physical environment. I first implemented the algorithm within Gazebo, a robotic simulation environment, on a similar model of the robot I intended to build. Afterwards, I built the robot and equipped it with a \ang{360} LiDAR for mapping and an Inertial Measurement Unit (IMU)/wheel encoders for localization. The physical robot was powered with a Raspberry Pi 4B and programmed using Python and the Robot Operating System 2 (ROS2) framework, which contains different algorithms and packages for robotics and data processing. 
\end{abstract}

\begin{IEEEkeywords}
\keywords
\end{IEEEkeywords}

\end{document}