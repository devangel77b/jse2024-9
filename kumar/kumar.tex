\documentclass[12pt,conference,onecolumn]{IEEEtran}

\title{{M}onmouth {C}ounty {E}ngineering traffic safety internship}
\author{\IEEEauthorblockN{Aadarsh Kumar}\IEEEauthorblockA{Science \& Engineering\\Manalapan High School\\Englishtown, NJ\\425akumar@frhsd.com}}
\date{June 16, 2025}

\newcommand{\keywords}{traffic safety, Monmouth County Engineering, civil engineering, internship, transportation systems, public sector, public works}

\usepackage{hyperref}
\makeatletter
\AtBeginDocument{
\hypersetup{%
pdftitle={\@title},
pdfauthor={Aadarsh Kumar},
pdfkeywords={\keywords}}}
\makeatother

\begin{document}
\maketitle 

\begin{abstract}
Civil engineering is a dynamic and evolving field that has to constantly adapt and change to accommodate new challenges and address the concerns of the government and its citizens to ensure the proper and safe function of all physical elements of the public sector. Monmouth County Engineering is a public works organization that designs, builds, and maintains the entirety of Monmouth County's transportation network and buildings. The Traffic Safety Division in particular deals with signals, signs, roads, and road markings. 

This internship has given me valuable experience in civil engineering and the various systems and methods engineers use to keep our roads safe. During my time at Monmouth County Engineering, I gained in-depth experience regarding the work and day-to-day functions and duties of a public works organization. My work included on-site inspections, signage designations, planning roadway changes, and striping estimates.
\end{abstract}

\begin{IEEEkeywords}
\keywords
\end{IEEEkeywords}

\end{document}